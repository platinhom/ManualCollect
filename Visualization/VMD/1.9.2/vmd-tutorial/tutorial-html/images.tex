\batchmode


\documentclass[letterpaper]{article}
\RequirePackage{ifthen}


\usepackage{html}
\usepackage{graphicx}
\usepackage{wrapfig}




\usepackage[dvips]{color}


\pagecolor[gray]{.7}

\usepackage[latin1]{inputenc}



\makeatletter
\AtBeginDocument{\makeatletter
\input /Scr/villa/ss/foo/RCS-check/vmd/vmd-tutorial.aux
\makeatother
}

\makeatletter
\count@=\the\catcode`\_ \catcode`\_=8 
\newenvironment{tex2html_wrap}{}{}%
\catcode`\<=12\catcode`\_=\count@
\newcommand{\providedcommand}[1]{\expandafter\providecommand\csname #1\endcsname}%
\newcommand{\renewedcommand}[1]{\expandafter\providecommand\csname #1\endcsname{}%
  \expandafter\renewcommand\csname #1\endcsname}%
\newcommand{\newedenvironment}[1]{\newenvironment{#1}{}{}\renewenvironment{#1}}%
\let\newedcommand\renewedcommand
\let\renewedenvironment\newedenvironment
\makeatother
\let\mathon=$
\let\mathoff=$
\ifx\AtBeginDocument\undefined \newcommand{\AtBeginDocument}[1]{}\fi
\newbox\sizebox
\setlength{\hoffset}{0pt}\setlength{\voffset}{0pt}
\addtolength{\textheight}{\footskip}\setlength{\footskip}{0pt}
\addtolength{\textheight}{\topmargin}\setlength{\topmargin}{0pt}
\addtolength{\textheight}{\headheight}\setlength{\headheight}{0pt}
\addtolength{\textheight}{\headsep}\setlength{\headsep}{0pt}
\setlength{\textwidth}{349pt}
\newwrite\lthtmlwrite
\makeatletter
\let\realnormalsize=\normalsize
\global\topskip=2sp
\def\preveqno{}\let\real@float=\@float \let\realend@float=\end@float
\def\@float{\let\@savefreelist\@freelist\real@float}
\def\liih@math{\ifmmode$\else\bad@math\fi}
\def\end@float{\realend@float\global\let\@freelist\@savefreelist}
\let\real@dbflt=\@dbflt \let\end@dblfloat=\end@float
\let\@largefloatcheck=\relax
\let\if@boxedmulticols=\iftrue
\def\@dbflt{\let\@savefreelist\@freelist\real@dbflt}
\def\adjustnormalsize{\def\normalsize{\mathsurround=0pt \realnormalsize
 \parindent=0pt\abovedisplayskip=0pt\belowdisplayskip=0pt}%
 \def\phantompar{\csname par\endcsname}\normalsize}%
\def\lthtmltypeout#1{{\let\protect\string \immediate\write\lthtmlwrite{#1}}}%
\newcommand\lthtmlhboxmathA{\adjustnormalsize\setbox\sizebox=\hbox\bgroup\kern.05em }%
\newcommand\lthtmlhboxmathB{\adjustnormalsize\setbox\sizebox=\hbox to\hsize\bgroup\hfill }%
\newcommand\lthtmlvboxmathA{\adjustnormalsize\setbox\sizebox=\vbox\bgroup %
 \let\ifinner=\iffalse \let\)\liih@math }%
\newcommand\lthtmlboxmathZ{\@next\next\@currlist{}{\def\next{\voidb@x}}%
 \expandafter\box\next\egroup}%
\newcommand\lthtmlmathtype[1]{\gdef\lthtmlmathenv{#1}}%
\newcommand\lthtmllogmath{\dimen0\ht\sizebox \advance\dimen0\dp\sizebox
  \ifdim\dimen0>.95\vsize
   \lthtmltypeout{%
*** image for \lthtmlmathenv\space is too tall at \the\dimen0, reducing to .95 vsize ***}%
   \ht\sizebox.95\vsize \dp\sizebox\z@ \fi
  \lthtmltypeout{l2hSize %
:\lthtmlmathenv:\the\ht\sizebox::\the\dp\sizebox::\the\wd\sizebox.\preveqno}}%
\newcommand\lthtmlfigureA[1]{\let\@savefreelist\@freelist
       \lthtmlmathtype{#1}\lthtmlvboxmathA}%
\newcommand\lthtmlpictureA{\bgroup\catcode`\_=8 \lthtmlpictureB}%
\newcommand\lthtmlpictureB[1]{\lthtmlmathtype{#1}\egroup
       \let\@savefreelist\@freelist \lthtmlhboxmathB}%
\newcommand\lthtmlpictureZ[1]{\hfill\lthtmlfigureZ}%
\newcommand\lthtmlfigureZ{\lthtmlboxmathZ\lthtmllogmath\copy\sizebox
       \global\let\@freelist\@savefreelist}%
\newcommand\lthtmldisplayA{\bgroup\catcode`\_=8 \lthtmldisplayAi}%
\newcommand\lthtmldisplayAi[1]{\lthtmlmathtype{#1}\egroup\lthtmlvboxmathA}%
\newcommand\lthtmldisplayB[1]{\edef\preveqno{(\theequation)}%
  \lthtmldisplayA{#1}\let\@eqnnum\relax}%
\newcommand\lthtmldisplayZ{\lthtmlboxmathZ\lthtmllogmath\lthtmlsetmath}%
\newcommand\lthtmlinlinemathA{\bgroup\catcode`\_=8 \lthtmlinlinemathB}
\newcommand\lthtmlinlinemathB[1]{\lthtmlmathtype{#1}\egroup\lthtmlhboxmathA
  \vrule height1.5ex width0pt }%
\newcommand\lthtmlinlineA{\bgroup\catcode`\_=8 \lthtmlinlineB}%
\newcommand\lthtmlinlineB[1]{\lthtmlmathtype{#1}\egroup\lthtmlhboxmathA}%
\newcommand\lthtmlinlineZ{\egroup\expandafter\ifdim\dp\sizebox>0pt %
  \expandafter\centerinlinemath\fi\lthtmllogmath\lthtmlsetinline}
\newcommand\lthtmlinlinemathZ{\egroup\expandafter\ifdim\dp\sizebox>0pt %
  \expandafter\centerinlinemath\fi\lthtmllogmath\lthtmlsetmath}
\newcommand\lthtmlindisplaymathZ{\egroup %
  \centerinlinemath\lthtmllogmath\lthtmlsetmath}
\def\lthtmlsetinline{\hbox{\vrule width.1em \vtop{\vbox{%
  \kern.1em\copy\sizebox}\ifdim\dp\sizebox>0pt\kern.1em\else\kern.3pt\fi
  \ifdim\hsize>\wd\sizebox \hrule depth1pt\fi}}}
\def\lthtmlsetmath{\hbox{\vrule width.1em\kern-.05em\vtop{\vbox{%
  \kern.1em\kern0.8 pt\hbox{\hglue.17em\copy\sizebox\hglue0.8 pt}}\kern.3pt%
  \ifdim\dp\sizebox>0pt\kern.1em\fi \kern0.8 pt%
  \ifdim\hsize>\wd\sizebox \hrule depth1pt\fi}}}
\def\centerinlinemath{%
  \dimen1=\ifdim\ht\sizebox<\dp\sizebox \dp\sizebox\else\ht\sizebox\fi
  \advance\dimen1by.5pt \vrule width0pt height\dimen1 depth\dimen1 
 \dp\sizebox=\dimen1\ht\sizebox=\dimen1\relax}

\def\lthtmlcheckvsize{\ifdim\ht\sizebox<\vsize 
  \ifdim\wd\sizebox<\hsize\expandafter\hfill\fi \expandafter\vfill
  \else\expandafter\vss\fi}%
\providecommand{\selectlanguage}[1]{}%
\makeatletter \tracingstats = 1 


\begin{document}
\pagestyle{empty}\thispagestyle{empty}\lthtmltypeout{}%
\lthtmltypeout{latex2htmlLength hsize=\the\hsize}\lthtmltypeout{}%
\lthtmltypeout{latex2htmlLength vsize=\the\vsize}\lthtmltypeout{}%
\lthtmltypeout{latex2htmlLength hoffset=\the\hoffset}\lthtmltypeout{}%
\lthtmltypeout{latex2htmlLength voffset=\the\voffset}\lthtmltypeout{}%
\lthtmltypeout{latex2htmlLength topmargin=\the\topmargin}\lthtmltypeout{}%
\lthtmltypeout{latex2htmlLength topskip=\the\topskip}\lthtmltypeout{}%
\lthtmltypeout{latex2htmlLength headheight=\the\headheight}\lthtmltypeout{}%
\lthtmltypeout{latex2htmlLength headsep=\the\headsep}\lthtmltypeout{}%
\lthtmltypeout{latex2htmlLength parskip=\the\parskip}\lthtmltypeout{}%
\lthtmltypeout{latex2htmlLength oddsidemargin=\the\oddsidemargin}\lthtmltypeout{}%
\makeatletter
\if@twoside\lthtmltypeout{latex2htmlLength evensidemargin=\the\evensidemargin}%
\else\lthtmltypeout{latex2htmlLength evensidemargin=\the\oddsidemargin}\fi%
\lthtmltypeout{}%
\makeatother
\setcounter{page}{1}
\onecolumn

% !!! IMAGES START HERE !!!



\newcounter{nitems}[subsection]%


%
\providecommand{\nitemold} [1]{\noindent \addtocounter{nitems}{1} \arabic{nitems}. #1 \newline}%


%
\providecommand{\descr} [1]{\noindent { #1 \newline} }%


%
\providecommand{\nitem} [1]{\addtocounter{nitems}{1} \begin{list}{\noindent \textbf{\arabic{nitems}}}{} \item #1 \end{list}}%


%
\providecommand{\sciencebox} [2]{
  \begin {center} 
  \fbox{
    %\htmlimage{scale=1.5}
    \begin{minipage}{.2\textwidth}
      \includegraphics[width=2.3 cm, height=2.3 cm]{pictures/tut0_science}
    \end{minipage}
    \begin{minipage}[r]{.75\textwidth}
      \noindent\small\textsf{\textbf{#1.} #2}
    \end{minipage}
  }
  \end {center}
}%


%
\providecommand{\infobox} [2]{
  \begin {center} 
  \framebox[\textwidth]{
    %\htmlimage{scale=1.5}
    \begin{minipage}{.2\textwidth}
      \includegraphics[width=2.5 cm, height=1.6 cm]{pictures/tut0_infobox}
   \end{minipage}
    \begin{minipage}[r]{.75\textwidth}
      \noindent\small\textsf{\textbf{#1.} #2}
    \end{minipage}
  }
  \end {center}
}%


%
\providecommand{\advancedmarcos} [2]{\begin{minipage}{.2\textwidth}
\fbox{\resizebox{2.5 cm}{#1 cm}{\includegraphics{pictures/tut0_vmdinfo}}}
\end{minipage}
\begin{minipage}[r]{.78\textwidth}
\begin{tabular}{|p{\textwidth}|}
\hline \rule{0pt}{2 ex }\\
{\small \textsf{ #2 \newline}}
\rule{0pt}{2ex}\\
\hline
\end{tabular}
\end{minipage}
}%


%
\providecommand{\button}{\sf}%


%
\providecommand{\menu}{\sf}%


%
\providecommand{\form}{}%


%
\providecommand{\file}{\tt}%


%
\providecommand{\tcl}{\tt}%


%
\providecommand{\infig}[1]{\textbf{#1}}%


%
\providecommand{\myfigure}[3]{
  \begin{figure}[ht!]
  \begin{center}
\par
\htmlimage{scale=2.0}
\par
\latex{
      \includegraphics[scale=0.5]{pictures/#1}
    }
    \end{center}
    \caption{#2}
    
  \end{figure}
}%


%
\providecommand{\twocolumns}[4]{
 %html
#1
\begin{figure}[h]  
    \begin{center}
     \htmlimage{scale=2.0}
        \includegraphics[scale=0.5]{pictures/#2}
    \end{center}
    \caption{#3}
    
 \end{figure}
}%


%
\providecommand{\tclcommandblock}[1]{
  \par
\begin{tabular}{ll}
    #1
  \end{tabular}
}%


%
\providecommand{\vmdcommanddef} [1]{
  \noindent \begin{center}
    \framebox[0.9\textwidth]{
    
  \par
\begin{tabular}{ll}
      
      #1
    
  \end{tabular}

  }
  \end{center}
}%

{\newpage\clearpage
\lthtmlfigureA{figure224}%
\begin{figure}\begin{center}
\htmlimage{scale=0}
\includegraphics[height=3in]{pictures/tut0_ubiquitin}
\end{center}
\end{figure}%
\lthtmlfigureZ
\lthtmlcheckvsize\clearpage}

{\newpage\clearpage
\lthtmlinlinemathA{tex2html_wrap_inline280}%
\fbox{
    \begin{minipage}[c]{\textwidth}
      \centering{\noindent\small{\small A current version of this tutorial is available at
\par
\htmladdnormallink {http://www.ks.uiuc.edu/Training/Tutorials/} {http://www.ks.uiuc.edu/Training/Tutorials/}}}
    \end{minipage}
  }%
\lthtmlinlinemathZ
\lthtmlcheckvsize\clearpage}

{\newpage\clearpage
\lthtmlinlinemathA{tex2html_wrap_inline339}%
\fbox{
    \begin{minipage}{.2\textwidth}
      \includegraphics[width=2.3 cm, height=2.3 cm]{pictures/tut0_science}
    \end{minipage}
    \begin{minipage}[r]{.75\textwidth}
      \noindent\small\textsf{\textbf{Ubiquitin.} This tutorial will focus on the visualization of \emph{ubiquitin} with VMD. Ubiquitin is a small protein of 76 amino acids, that is believed to be present in all eukaryotic cells. It is one of the most conserved of all eukaryotic proteins (the first 74 amino acids form a structure that is identical in insects, trout, bovines and human) and it has been identified in the nucleus, cytoplasm and on the cell-surface. It's primary role is in protein degradation, where it acts as a tag for intracellular proteolysis.}
    \end{minipage}
  }%
\lthtmlinlinemathZ
\lthtmlcheckvsize\clearpage}

\stepcounter{section}
\stepcounter{subsection}
\addtocounter{nitems}{1}
{\newpage\clearpage
\lthtmlinlinemathA{tex2html_wrap_inline1430}%
$\rightarrow$%
\lthtmlinlinemathZ
\lthtmlcheckvsize\clearpage}

\addtocounter{nitems}{1}
{\newpage\clearpage
\lthtmlfigureA{figure656}%
\begin{figure}    \begin{center}
     \htmlimage{scale=2.0}
        \includegraphics[scale=0.5]{pictures/tut_unit01_001}
    \end{center}
    
 \end{figure}%
\lthtmlfigureZ
\lthtmlcheckvsize\clearpage}

{\newpage\clearpage
\lthtmlpictureA{tex2html_wrap1506}%
\framebox[\textwidth]{
    \begin{minipage}{.2\textwidth}
      \includegraphics[width=2.5 cm, height=1.6 cm]{pictures/tut0_infobox}
   \end{minipage}
    \begin{minipage}[r]{.75\textwidth}
      \noindent\small\textsf{\textbf{Webpdb.} VMD can download a pdb file from the Protein Data Bank 
if a network connection is available. Just type the four letter code 
of the protein in the {\sf File Name} text entry of the {Molecule
File Browser} window and press the {\sf Load} button. 
VMD will download it automatically.}
    \end{minipage}
  }%
\lthtmlpictureZ
\lthtmlcheckvsize\clearpage}

{\newpage\clearpage
\lthtmlinlinemathA{tex2html_wrap_inline1510}%
\fbox{
    \begin{minipage}{.2\textwidth}
      \includegraphics[width=2.3 cm, height=2.3 cm]{pictures/tut0_science}
    \end{minipage}
    \begin{minipage}[r]{.75\textwidth}
      \noindent\small\textsf{\textbf{Coordinates file.} The file {\tt 1UBQ.pdb} corresponds to 
the X-ray structure of ubiquitin refined at 1.8~\AA\ resolution provided 
by Senadhi Vijay-Kumar, 
Charles E. Bugg and William J. Cook, J. Mol. Biol. (1987) \textbf{194}, 
531. Note that the protein is sorrounded by 58 water molecules, and 
that hydrogen atoms are not included.}
    \end{minipage}
  }%
\lthtmlinlinemathZ
\lthtmlcheckvsize\clearpage}

\stepcounter{subsection}
\addtocounter{nitems}{1}
{\newpage\clearpage
\lthtmlfigureA{figure736}%
\begin{figure}    \begin{center}
     \htmlimage{scale=2.0}
        \includegraphics[scale=0.5]{pictures/tut_rotations}
    \end{center}
    
 \end{figure}%
\lthtmlfigureZ
\lthtmlcheckvsize\clearpage}

\addtocounter{nitems}{1}
\addtocounter{nitems}{1}
\addtocounter{nitems}{1}
{\newpage\clearpage
\lthtmlfigureA{figure766}%
\begin{figure}    \begin{center}
     \htmlimage{scale=2.0}
        \includegraphics[scale=0.5]{pictures/tut_unit01_002}
    \end{center}
    
 \end{figure}%
\lthtmlfigureZ
\lthtmlcheckvsize\clearpage}

\addtocounter{nitems}{1}
{\newpage\clearpage
\lthtmlpictureA{tex2html_wrap1560}%
\framebox[\textwidth]{
    \begin{minipage}{.2\textwidth}
      \includegraphics[width=2.5 cm, height=1.6 cm]{pictures/tut0_infobox}
   \end{minipage}
    \begin{minipage}[r]{.75\textwidth}
      \noindent\small\textsf{\textbf{Mouse modes.} Note that each mouse mode has its own characteristic 
cursor and its own shortcut key ({\tt r}: Rotate, {\tt t}: Translate, 
{\tt s}: Scale) that could be used instead of the {\sf Mouse} menu. 
(Be sure to have the 
{OpenGL Display} window active when using the shortcuts.) Additional
information can be found in the \htmladdnormallink{VMD user's guide.}{http://www.ks.uiuc.edu/Research/vmd/current/ug/node27.html}}
    \end{minipage}
  }%
\lthtmlpictureZ
\lthtmlcheckvsize\clearpage}

\addtocounter{nitems}{1}
\addtocounter{nitems}{1}
\stepcounter{subsection}
\addtocounter{nitems}{1}
\addtocounter{nitems}{1}
\addtocounter{nitems}{1}
\addtocounter{nitems}{1}
{\newpage\clearpage
\lthtmlfigureA{figure857}%
\begin{figure}    \begin{center}
     \htmlimage{scale=2.0}
        \includegraphics[scale=0.5]{pictures/tut_unit01_004}
    \end{center}
    
 \end{figure}%
\lthtmlfigureZ
\lthtmlcheckvsize\clearpage}

\addtocounter{nitems}{1}
\addtocounter{nitems}{1}
\addtocounter{nitems}{1}
{\newpage\clearpage
\lthtmlpictureA{tex2html_wrap1650}%
\framebox[\textwidth]{
    \begin{minipage}{.2\textwidth}
      \includegraphics[width=2.5 cm, height=1.6 cm]{pictures/tut0_infobox}
   \end{minipage}
    \begin{minipage}[r]{.75\textwidth}
      \noindent\small\textsf{\textbf{More representations.} Other interesting representations are {\sf CPK} and 
{\sf Licorice}. In the first one, like in old chemistry ball \& stick kits, each atom is represented by a sphere 
and each bond is represented by a cylinder. (Radius and resolution of both 
the sphere and the cylinder can be modified independently.) The Licorice 
drawing method (widely used) also represents each atom as a sphere and each 
bond as a cylinder, but the sphere radius cannot be modified independently.}
    \end{minipage}
  }%
\lthtmlpictureZ
\lthtmlcheckvsize\clearpage}

\addtocounter{nitems}{1}
\addtocounter{nitems}{1}
{\newpage\clearpage
\lthtmlinlinemathA{tex2html_wrap_inline1436}%
$\beta$%
\lthtmlinlinemathZ
\lthtmlcheckvsize\clearpage}

\addtocounter{nitems}{1}
\addtocounter{nitems}{1}
{\newpage\clearpage
\lthtmlfigureA{figure998}%
\begin{figure}  \begin{center}
\par
\htmlimage{scale=2.0}
\par
\latex{
      \includegraphics[scale=0.5]{pictures/tut_licotube}
    }
    \end{center}
    
  \end{figure}%
\lthtmlfigureZ
\lthtmlcheckvsize\clearpage}

{\newpage\clearpage
\lthtmlinlinemathA{tex2html_wrap_inline1704}%
\fbox{
    \begin{minipage}{.2\textwidth}
      \includegraphics[width=2.3 cm, height=2.3 cm]{pictures/tut0_science}
    \end{minipage}
    \begin{minipage}[r]{.75\textwidth}
      \noindent\small\textsf{\textbf{Structure of ubiquitin.} 
Ubiquitin has three and one half turns of $\alpha$-helix (residues
23 to 34, three of them 
hydrophobic), one short piece of $3_{10}$-helix (residues 56 to 59) and a mixed 
$\beta$\  sheet with five strands (residues 1 to 7, 10 to 17, 40 to 45, 48 to 50, and 64 to 72) 
and seven reverse turns. VMD calculates
the secondary structure using STRIDE, which uses an heuristic
algorithm that in this case shows only four of the five
$\beta$\  strands.}
    \end{minipage}
  }%
\lthtmlinlinemathZ
\lthtmlcheckvsize\clearpage}

\stepcounter{subsection}
\addtocounter{nitems}{1}
\addtocounter{nitems}{1}
\stepcounter{subsection}
\addtocounter{nitems}{1}
\addtocounter{nitems}{1}
\addtocounter{nitems}{1}
\addtocounter{nitems}{1}
\addtocounter{nitems}{1}
\addtocounter{nitems}{1}
\addtocounter{nitems}{1}
{\newpage\clearpage
\lthtmlfigureA{figure1093}%
\begin{figure}    \begin{center}
     \htmlimage{scale=2.0}
        \includegraphics[scale=0.5]{pictures/tut_unit01_005}
    \end{center}
    
 \end{figure}%
\lthtmlfigureZ
\lthtmlcheckvsize\clearpage}

\addtocounter{nitems}{1}
{\newpage\clearpage
\lthtmlinlinemathA{tex2html_wrap_inline1452}%
$\alpha$%
\lthtmlinlinemathZ
\lthtmlcheckvsize\clearpage}

\stepcounter{subsection}
\addtocounter{nitems}{1}
\addtocounter{nitems}{1}
\addtocounter{nitems}{1}
{\newpage\clearpage
\lthtmlfigureA{figure1156}%
\begin{figure}    \begin{center}
     \htmlimage{scale=2.0}
        \includegraphics[scale=0.5]{pictures/tut_unit01_006}
    \end{center}
    
 \end{figure}%
\lthtmlfigureZ
\lthtmlcheckvsize\clearpage}

\addtocounter{nitems}{1}
\addtocounter{nitems}{1}
\addtocounter{nitems}{1}
\stepcounter{subsection}
\addtocounter{nitems}{1}
\addtocounter{nitems}{1}
\addtocounter{nitems}{1}
\addtocounter{nitems}{1}
{\newpage\clearpage
\lthtmlfigureA{figure1254}%
\begin{figure}    \begin{center}
     \htmlimage{scale=2.0}
        \includegraphics[scale=0.5]{pictures/tut_unit01_008}
    \end{center}
    
 \end{figure}%
\lthtmlfigureZ
\lthtmlcheckvsize\clearpage}

\addtocounter{nitems}{1}
{\newpage\clearpage
\lthtmlinlinemathA{tex2html_wrap_inline1986}%
\fbox{
    \begin{minipage}{.2\textwidth}
      \includegraphics[width=2.3 cm, height=2.3 cm]{pictures/tut0_science}
    \end{minipage}
    \begin{minipage}[r]{.75\textwidth}
      \noindent\small\textsf{\textbf{Relevance of Lysines.} Polyubiquitin chains can be linked by a peptide bond between C- and N-termini 
or through linkages involving lysines
48, 63, 11 or 29 (the ones that you selected using the sequence window).
Different linked chains have different properties
that are related to the functionality of the chain.}
    \end{minipage}
  }%
\lthtmlinlinemathZ
\lthtmlcheckvsize\clearpage}

\stepcounter{subsection}
\addtocounter{nitems}{1}
\addtocounter{nitems}{1}
\addtocounter{nitems}{1}
\addtocounter{nitems}{1}
\addtocounter{nitems}{1}
\addtocounter{nitems}{1}
\addtocounter{nitems}{1}
\addtocounter{nitems}{1}
\addtocounter{nitems}{1}
\stepcounter{section}
\addtocounter{nitems}{1}
\stepcounter{subsection}
\addtocounter{nitems}{1}
\addtocounter{nitems}{1}
{\newpage\clearpage
\lthtmlpictureA{tex2html_wrap3169}%
\framebox[\textwidth]{
    \begin{minipage}{.2\textwidth}
      \includegraphics[width=2.5 cm, height=1.6 cm]{pictures/tut0_infobox}
   \end{minipage}
    \begin{minipage}[r]{.75\textwidth}
      \noindent\small\textsf{\textbf{Coordinate files and structure files.} To save space, simulation output files usually contain only the atom coordinates and do not store unchanging information such as the atom types, atom charge, segment names and bonds. The latter information is stored in a separate ``structure" file (\emph{e.g.}, a PSF file). To visualize the results of a simulation, you need to merge both a structure and a coordinate file into the same molecule.}
    \end{minipage}
  }%
\lthtmlpictureZ
\lthtmlcheckvsize\clearpage}

\addtocounter{nitems}{1}
\addtocounter{nitems}{1}
\stepcounter{subsection}
\addtocounter{nitems}{1}
{\newpage\clearpage
\lthtmlfigureA{figure2470}%
\begin{figure}  \begin{center}
\par
\htmlimage{scale=2.0}
\par
\latex{
      \includegraphics[scale=0.5]{pictures/tut2_main_2mols}
    }
    \end{center}
    
  \end{figure}%
\lthtmlfigureZ
\lthtmlcheckvsize\clearpage}

\addtocounter{nitems}{1}
\addtocounter{nitems}{1}
\addtocounter{nitems}{1}
\addtocounter{nitems}{1}
\addtocounter{nitems}{1}
\stepcounter{subsection}
{\newpage\clearpage
\lthtmlpictureA{tex2html_wrap3281}%
\framebox[\textwidth]{
    \begin{minipage}{.2\textwidth}
      \includegraphics[width=2.5 cm, height=1.6 cm]{pictures/tut0_infobox}
   \end{minipage}
    \begin{minipage}[r]{.75\textwidth}
      \noindent\small\textsf{\textbf{The Tcl/Tk scripting language.} Tcl is a rich language that contains many other features and commands than those seen above, in addition to the typical conditional and looping expressions. Tk is an extension to Tcl that permits the writing of user interfaces with windows and buttons, \emph{etc}. More information and docs about the Tcl/Tk language can be found at \htmladdnormallink{http://www.tcl.tk/doc}{http://www.tcl.tk/doc/}. }
    \end{minipage}
  }%
\lthtmlpictureZ
\lthtmlcheckvsize\clearpage}

\addtocounter{nitems}{1}
{\newpage\clearpage
\lthtmlfigureA{figure2562}%
\begin{figure}  \begin{center}
\par
\htmlimage{scale=2.0}
\par
\latex{
      \includegraphics[scale=0.5]{pictures/tut2_tkcon}
    }
    \end{center}
    
  \end{figure}%
\lthtmlfigureZ
\lthtmlcheckvsize\clearpage}

{\newpage\clearpage
\lthtmlpictureA{tex2html_wrap3311}%
\framebox[0.9\textwidth]{
    
  \par
\begin{tabular}{ll}
      
      
    {\tt set} {\it variable} {\it value} & -- sets the value of {\it variable}  \\
    {\tt puts} {\tt \$}{\it variable} & -- prints out the value of {\it variable}

    
  \end{tabular}

  }%
\lthtmlpictureZ
\lthtmlcheckvsize\clearpage}

\addtocounter{nitems}{1}
{\newpage\clearpage
\lthtmlpictureA{tex2html_wrap3337}%
\framebox[0.9\textwidth]{
    
  \par
\begin{tabular}{ll}
      
      
    {\tt expr} {\it expression} & -- evaluates a mathematical expression 

    
  \end{tabular}

  }%
\lthtmlpictureZ
\lthtmlcheckvsize\clearpage}

\addtocounter{nitems}{1}
{\newpage\clearpage
\lthtmlpictureA{tex2html_wrap3359}%
\framebox[0.9\textwidth]{
    
  \par
\begin{tabular}{ll}
      
      
    {\tt [{\rm \it expr.}]} & -- represents the result of the expression inside the brackets 

    
  \end{tabular}

  }%
\lthtmlpictureZ
\lthtmlcheckvsize\clearpage}

\addtocounter{nitems}{1}
\stepcounter{subsection}
\addtocounter{nitems}{1}
{\newpage\clearpage
\lthtmlpictureA{tex2html_wrap3395}%
\framebox[\textwidth]{
    \begin{minipage}{.2\textwidth}
      \includegraphics[width=2.5 cm, height=1.6 cm]{pictures/tut0_infobox}
   \end{minipage}
    \begin{minipage}[r]{.75\textwidth}
      \noindent\small\textsf{\textbf{The PDB B-factor field.} The ``B" field of a PDB file typically stores the ``temperature factor" for a crystal structure and is read into VMD's ``Beta" field. Since we are not currently interested in this information, we can recycle this field to store our own numerical values. VMD has a ``Beta" coloring mode, which you will soon use, and which colors atoms according to their B-factors. Thus, by replacing the Beta values for various atoms, you can control the color in which they are drawn. This is very useful when you want to show a property of the system that you have computed and which is not supported out-of-the-box by VMD.}
    \end{minipage}
  }%
\lthtmlpictureZ
\lthtmlcheckvsize\clearpage}

\addtocounter{nitems}{1}
{\newpage\clearpage
\lthtmlpictureA{tex2html_wrap3415}%
\framebox[0.9\textwidth]{
    
  \par
\begin{tabular}{ll}
      
      
    {\tt atomselect} {\it molid} {\it selection} & -- creates a new atom selection

    
  \end{tabular}

  }%
\lthtmlpictureZ
\lthtmlcheckvsize\clearpage}

\addtocounter{nitems}{1}
\addtocounter{nitems}{1}
\addtocounter{nitems}{1}
\addtocounter{nitems}{1}
\addtocounter{nitems}{1}
{\newpage\clearpage
\lthtmlpictureA{tex2html_wrap3481}%
\framebox[\textwidth]{
    \begin{minipage}{.2\textwidth}
      \includegraphics[width=2.5 cm, height=1.6 cm]{pictures/tut0_infobox}
   \end{minipage}
    \begin{minipage}[r]{.75\textwidth}
      \noindent\small\textsf{\textbf{Examples of atomic properties.} You can get and set many atomic properties using atom selections, including segment, residue and atom names, position (x, y and z), charge, mass, occupancy and radius, just to name a few.}
    \end{minipage}
  }%
\lthtmlpictureZ
\lthtmlcheckvsize\clearpage}

\addtocounter{nitems}{1}
{\newpage\clearpage
\lthtmlfigureA{figure2800}%
\begin{figure}  \begin{center}
\par
\htmlimage{scale=2.0}
\par
\latex{
      \includegraphics[scale=0.5]{pictures/tut2_hydrophobicity}
    }
    \end{center}
    
  \end{figure}%
\lthtmlfigureZ
\lthtmlcheckvsize\clearpage}

{\newpage\clearpage
\lthtmlinlinemathA{tex2html_wrap_inline3505}%
\fbox{
    \begin{minipage}{.2\textwidth}
      \includegraphics[width=2.3 cm, height=2.3 cm]{pictures/tut0_science}
    \end{minipage}
    \begin{minipage}[r]{.75\textwidth}
      \noindent\small\textsf{\textbf{Hydrophobic residues.} As you probably noticed in your rendering of ubiquitin, the hydrophobic residues are almost exclusively contained in the inner core of the protein. This is typical for small water-soluble proteins. As the protein folds, the hydrophylic residues will have a tendency to stay at the water interface, while the hydrophobic residues are pushed together and play a structural role. This helps the protein achieve proper folding and increases it's stability.}
    \end{minipage}
  }%
\lthtmlinlinemathZ
\lthtmlcheckvsize\clearpage}

\addtocounter{nitems}{1}
\addtocounter{nitems}{1}
\addtocounter{nitems}{1}
\stepcounter{subsection}
\addtocounter{nitems}{1}
\addtocounter{nitems}{1}
\addtocounter{nitems}{1}
\addtocounter{nitems}{1}
{\newpage\clearpage
\lthtmlpictureA{tex2html_wrap3625}%
\framebox[0.9\textwidth]{
    
  \par
\begin{tabular}{ll}
      
      
    {\tt measure fit} {\it atomsel1} {\it atomsel2} & -- calculates the best fit matrix 

    
  \end{tabular}

  }%
\lthtmlpictureZ
\lthtmlcheckvsize\clearpage}

\addtocounter{nitems}{1}
\addtocounter{nitems}{1}
{\newpage\clearpage
\lthtmlfigureA{figure2977}%
\begin{figure}  \begin{center}
\par
\htmlimage{scale=2.0}
\par
\latex{
      \includegraphics[scale=0.5]{pictures/tut2_ubiquitin_aligned}
    }
    \end{center}
    
  \end{figure}%
\lthtmlfigureZ
\lthtmlcheckvsize\clearpage}

\stepcounter{subsection}
{\newpage\clearpage
\lthtmlpictureA{tex2html_wrap3669}%
\framebox[0.9\textwidth]{
    
  \par
\begin{tabular}{ll}
      
      
  {\tt source} {\it file} & -- runs a script from a text file 

    
  \end{tabular}

  }%
\lthtmlpictureZ
\lthtmlcheckvsize\clearpage}

\addtocounter{nitems}{1}
\addtocounter{nitems}{1}
\addtocounter{nitems}{1}
{\newpage\clearpage
\lthtmlfigureA{figure3042}%
\begin{figure}  \begin{center}
\par
\htmlimage{scale=2.0}
\par
\latex{
      \includegraphics[scale=0.5]{pictures/tut2_colorrange}
    }
    \end{center}
    
  \end{figure}%
\lthtmlfigureZ
\lthtmlcheckvsize\clearpage}

\addtocounter{nitems}{1}
{\newpage\clearpage
\lthtmlfigureA{figure3065}%
\begin{figure}  \begin{center}
\par
\htmlimage{scale=2.0}
\par
\latex{
      \includegraphics[scale=0.5]{pictures/tut2_colorscale}
    }
    \end{center}
    
  \end{figure}%
\lthtmlfigureZ
\lthtmlcheckvsize\clearpage}

\addtocounter{nitems}{1}
\addtocounter{nitems}{1}
{\newpage\clearpage
\lthtmlfigureA{figure3098}%
\begin{figure}  \begin{center}
\par
\htmlimage{scale=2.0}
\par
\latex{
      \includegraphics[scale=0.5]{pictures/tut2_ubiquitin_colordisp}
    }
    \end{center}
    
  \end{figure}%
\lthtmlfigureZ
\lthtmlcheckvsize\clearpage}


%
\providecommand{\mybutton}[1]{\raisebox{-0.5ex}{\includegraphics[scale=0.5]{pictures/#1}}}%

\stepcounter{section}
\stepcounter{subsection}
\addtocounter{nitems}{1}
\addtocounter{nitems}{1}
\addtocounter{nitems}{1}
\addtocounter{nitems}{1}
\addtocounter{nitems}{1}
\addtocounter{nitems}{1}
{\newpage\clearpage
\lthtmlinlinemathA{tex2html_wrap_inline5244}%
\fbox{
    \begin{minipage}{.2\textwidth}
      \includegraphics[width=2.3 cm, height=2.3 cm]{pictures/tut0_science}
    \end{minipage}
    \begin{minipage}[r]{.75\textwidth}
      \noindent\small\textsf{\textbf{Elastic properties of ubiquitin.} Ubiquitin has many functions in the cell. It is currently believed that some of these functions depend on the elastic properties of
ubiquitin. \\\\Only certain proteins are known to show elastic
properties. Examples of these molecules are titin, which acts like a
``spring'' in the muscle. The elastic properties of these molecules are due
to hydrogen bonding between residues in $\beta$~strands of this molecules,
like the ones in ubiquitin.\\\\The trajectory you just loaded is a
simulation of an AFM (Atomic Force Microscopy) experiment pulling on a
single ubiquitin molecule. We will look at the behavior of the protein as
it unfolds while being pulled form one end.\\\\In Molecular Dynamics,
before starting a simulation like the one shown for the AFM experiment, one
first needs to perform the equilibration of the protein. You will have a
chance to look at such an equilibration trajectory later in this tutorial.  }
    \end{minipage}
  }%
\lthtmlinlinemathZ
\lthtmlcheckvsize\clearpage}

\stepcounter{subsection}
{\newpage\clearpage
\lthtmlpictureA{tex2html_wrap5258}%
\framebox[0.9\textwidth]{
    
  \par
\begin{tabular}{ll}
      
       {\tt atomselect macro } {\it name} {\it selection} & --
creates a macro for {\it selection} 
    
  \end{tabular}

  }%
\lthtmlpictureZ
\lthtmlcheckvsize\clearpage}

\addtocounter{nitems}{1}
\addtocounter{nitems}{1}
\addtocounter{nitems}{1}
\addtocounter{nitems}{1}
\addtocounter{nitems}{1}
\addtocounter{nitems}{1}
\addtocounter{nitems}{1}
\addtocounter{nitems}{1}
\addtocounter{nitems}{1}
\addtocounter{nitems}{1}
\addtocounter{nitems}{1}
\addtocounter{nitems}{1}
\addtocounter{nitems}{1}
\addtocounter{nitems}{1}
\addtocounter{nitems}{1}
{\newpage\clearpage
\lthtmlfigureA{figure4333}%
\begin{figure}    \begin{center}
     \htmlimage{scale=2.0}
        \includegraphics[scale=0.5]{pictures/tut_macros_gui}
    \end{center}
    
 \end{figure}%
\lthtmlfigureZ
\lthtmlcheckvsize\clearpage}

\addtocounter{nitems}{1}
{\newpage\clearpage
\lthtmlpictureA{tex2html_wrap5454}%
\framebox[\textwidth]{
    \begin{minipage}{.2\textwidth}
      \includegraphics[width=2.5 cm, height=1.6 cm]{pictures/tut0_infobox}
   \end{minipage}
    \begin{minipage}[r]{.75\textwidth}
      \noindent\small\textsf{\textbf{Macros.} You can create several macros and store them in a VMD preferences
file {\tt .vmdrc} in your home directory (Windows uses the file {\tt 
vmd.rc}). VMD will look for this file upon startup and will recognize all
your macros. For more information of the VMD startup files, refer to the
VMD user's guide.}
    \end{minipage}
  }%
\lthtmlpictureZ
\lthtmlcheckvsize\clearpage}

{\newpage\clearpage
\lthtmlinlinemathA{tex2html_wrap_inline5460}%
\fbox{
    \begin{minipage}{.2\textwidth}
      \includegraphics[width=2.3 cm, height=2.3 cm]{pictures/tut0_science}
    \end{minipage}
    \begin{minipage}[r]{.75\textwidth}
      \noindent\small\textsf{\textbf{Hydrogen Bonds.} The H-bonds give stability to proteins in several ways. The H-bonds
between $\beta$~sheet backbone atoms are a trademark of elastic proteins;
they give these proteins stability and their formation and rupture under
mechanical stress confers them their elastic properties. }
    \end{minipage}
  }%
\lthtmlinlinemathZ
\lthtmlcheckvsize\clearpage}

\addtocounter{nitems}{1}
\addtocounter{nitems}{1}
{\newpage\clearpage
\lthtmlpictureA{tex2html_wrap5484}%
\framebox[\textwidth]{
    \begin{minipage}{.2\textwidth}
      \includegraphics[width=2.5 cm, height=1.6 cm]{pictures/tut0_infobox}
   \end{minipage}
    \begin{minipage}[r]{.75\textwidth}
      \noindent\small\textsf{\textbf{Reusing saved states.} When you have a saved state for a system, you can use it later
for viewing different coordinates (PDB or a different trajectory), if these
have the same PSF file. You can do this by following these steps:\\
\begin{itemize}
\item Load your saved state file: {\sf File~$\rightarrow$~Load State...}
\item Select the file you saved before (i.e. {\tt nice-ubiquitin.vmd})
\item Go to {\sf Molecule~$\rightarrow$~Delete Frames...}. You need to
delete all frames currently loaded.The default options will do that. Click
on the {\sf Delete} button.
\item Load your new file {\tt yournewtrajectory.dcd} into the PSF in {\sf 
File~$\rightarrow$~Load Data Into Molecule}.
\end{itemize}
  }
    \end{minipage}
  }%
\lthtmlpictureZ
\lthtmlcheckvsize\clearpage}

{\newpage\clearpage
\lthtmlfigureA{figure4501}%
\begin{figure}  \begin{center}
\par
\htmlimage{scale=2.0}
\par
\latex{
      \includegraphics[scale=0.5]{pictures/tut_traj_nice}
    }
    \end{center}
    
  \end{figure}%
\lthtmlfigureZ
\lthtmlcheckvsize\clearpage}

\stepcounter{subsection}
\addtocounter{nitems}{1}
{\newpage\clearpage
\lthtmlfigureA{figure4529}%
\begin{figure}  \begin{center}
\par
\htmlimage{scale=2.0}
\par
\latex{
      \includegraphics[scale=0.5]{pictures/tut_animate}
    }
    \end{center}
    
  \end{figure}%
\lthtmlfigureZ
\lthtmlcheckvsize\clearpage}

\addtocounter{nitems}{1}
\addtocounter{nitems}{1}
{\newpage\clearpage
\lthtmlinlinemathA{tex2html_wrap_inline5534}%
\fbox{
    \begin{minipage}{.2\textwidth}
      \includegraphics[width=2.3 cm, height=2.3 cm]{pictures/tut0_science}
    \end{minipage}
    \begin{minipage}[r]{.75\textwidth}
      \noindent\small\textsf{\textbf{Minimizing free energy.}  Note how the shape of the water box changes from a cube to a
sphere. To minimize free energy, the water adopts the configuration that
exposes less surface to the vacm, that is, a sphere! Check how fast the
water comes to this configuration (Each frame step corresponds to 10 ps. )}
    \end{minipage}
  }%
\lthtmlinlinemathZ
\lthtmlcheckvsize\clearpage}

\addtocounter{nitems}{1}
{\newpage\clearpage
\lthtmlinlinemathA{tex2html_wrap_inline5544}%
\fbox{
    \begin{minipage}{.2\textwidth}
      \includegraphics[width=2.3 cm, height=2.3 cm]{pictures/tut0_science}
    \end{minipage}
    \begin{minipage}[r]{.75\textwidth}
      \noindent\small\textsf{\textbf{Breaking H-Bonds.} Look at the two yellow $\beta$~strands in the middle. They are
connected to each other with H-bonds.  Note how they are the first
$\beta$~strands to separate, this means their H-bonds are the first to
break. We will talk some more about the relevance of this phenomenon
later.}
    \end{minipage}
  }%
\lthtmlinlinemathZ
\lthtmlcheckvsize\clearpage}

\addtocounter{nitems}{1}
\addtocounter{nitems}{1}
{\newpage\clearpage
\lthtmlpictureA{tex2html_wrap5570}%
\framebox[\textwidth]{
    \begin{minipage}{.2\textwidth}
      \includegraphics[width=2.5 cm, height=1.6 cm]{pictures/tut0_infobox}
   \end{minipage}
    \begin{minipage}[r]{.75\textwidth}
      \noindent\small\textsf{\textbf{Looping styles.} When playing animations, you can choose between 3 looping styles
in the {Style Chooser}. These are ``Once'', ``Loop'' and ``Rock''.  A
nice trick:\\
  \begin{itemize}
    \item Go to the beggining of the trajectory (frame 0).
    \item Set the step number to the total number of frames in your
    trajectory (100).
    \item Set the style to ``Rock''.
    \item Set the speed to lowest.
    \item Play.
    \end{itemize}
    You can see now a comparison between the first and the last frame of
    your trajectory. This trick is better than clicking on the {\sf 
    Start} and {\sf End} buttons.}
    \end{minipage}
  }%
\lthtmlpictureZ
\lthtmlcheckvsize\clearpage}

{\newpage\clearpage
\lthtmlpictureA{tex2html_wrap5576}%

% latex2html id marker 5576
\framebox[\textwidth]{
    \begin{minipage}{.2\textwidth}
      \includegraphics[width=2.5 cm, height=1.6 cm]{pictures/tut0_infobox}
   \end{minipage}
    \begin{minipage}[r]{.75\textwidth}
      \noindent\small\textsf{\textbf{Recalculate secondary structure.} You can ask VMD to recalculate the secondary structure for a new frame using STRIDE. This was done for Fig.~\ref{fig:tut_traj_nice}, by typing the command {\tt mol ssrecalc top} in the Tk Console, while in frame 28.}
    \end{minipage}
  }%
\lthtmlpictureZ
\lthtmlcheckvsize\clearpage}

\stepcounter{subsection}
\addtocounter{nitems}{1}
\addtocounter{nitems}{1}
{\newpage\clearpage
\lthtmlinlinemathA{tex2html_wrap_inline5606}%
\fbox{
    \begin{minipage}{.2\textwidth}
      \includegraphics[width=2.3 cm, height=2.3 cm]{pictures/tut0_science}
    \end{minipage}
    \begin{minipage}[r]{.75\textwidth}
      \noindent\small\textsf{\textbf{Residue K48.} As you learned in Unit 1, polyubiquitin chains can be linked by a connection between the C terminus of one ubiquitin molecule and residue K48 of the next. The simulation then mimic the effect of pulling on the C terminus with this kind of linkage.}
    \end{minipage}
  }%
\lthtmlinlinemathZ
\lthtmlcheckvsize\clearpage}

\addtocounter{nitems}{1}
\addtocounter{nitems}{1}
{\newpage\clearpage
\lthtmlpictureA{tex2html_wrap5632}%
\framebox[\textwidth]{
    \begin{minipage}{.2\textwidth}
      \includegraphics[width=2.5 cm, height=1.6 cm]{pictures/tut0_infobox}
   \end{minipage}
    \begin{minipage}[r]{.75\textwidth}
      \noindent\small\textsf{\textbf{PDB and VMD atom numbering.} Note that the atom number of these atoms in the pdb file is 771
and 1243. VMD starts counting the {\tt index} from zero, so the text in the
Representation should be the numbers that VMD understands. This is only the
case for index, since VMD does not read them from the PDB file. Other
keywords, such as residue, are consistent with the PDB file.}
    \end{minipage}
  }%
\lthtmlpictureZ
\lthtmlcheckvsize\clearpage}

\addtocounter{nitems}{1}
\addtocounter{nitems}{1}
{\newpage\clearpage
\lthtmlfigureA{figure4736}%
\begin{figure}  \begin{center}
\par
\htmlimage{scale=2.0}
\par
\latex{
      \includegraphics[scale=0.5]{pictures/tut_bond_dist}
    }
    \end{center}
    
  \end{figure}%
\lthtmlfigureZ
\lthtmlcheckvsize\clearpage}

{\newpage\clearpage
\lthtmlpictureA{tex2html_wrap5660}%
\framebox[\textwidth]{
    \begin{minipage}{.2\textwidth}
      \includegraphics[width=2.5 cm, height=1.6 cm]{pictures/tut0_infobox}
   \end{minipage}
    \begin{minipage}[r]{.75\textwidth}
      \noindent\small\textsf{\textbf{Labels.} The shortcut keys for labels are {\tt 1}: Atoms and {\tt 2}:
  Bonds. You can use these instead of the {\sf Mouse} menu. Be sure the Open GL display window is active when using these shortcuts. }
    \end{minipage}
  }%
\lthtmlpictureZ
\lthtmlcheckvsize\clearpage}

\addtocounter{nitems}{1}
{\newpage\clearpage
\lthtmlfigureA{figure4772}%
\begin{figure}  \begin{center}
\par
\htmlimage{scale=2.0}
\par
\latex{
      \includegraphics[scale=0.5]{pictures/tut_label_window}
    }
    \end{center}
    
  \end{figure}%
\lthtmlfigureZ
\lthtmlcheckvsize\clearpage}

\addtocounter{nitems}{1}
\addtocounter{nitems}{1}
{\newpage\clearpage
\lthtmlpictureA{tex2html_wrap5706}%
\framebox[\textwidth]{
    \begin{minipage}{.2\textwidth}
      \includegraphics[width=2.5 cm, height=1.6 cm]{pictures/tut0_infobox}
   \end{minipage}
    \begin{minipage}[r]{.75\textwidth}
      \noindent\small\textsf{\textbf{xmgrace.} VMD uses the 2-D plotting program xmgrace. This program is
  available from  http://plasma-gate.weizmann.ac.il/Grace/. In Mac OSX, xmgrace
  runs in the X11 environment. Make sure X11 is running in order for xmgrace to
  start. }
    \end{minipage}
  }%
\lthtmlpictureZ
\lthtmlcheckvsize\clearpage}

\addtocounter{nitems}{1}
{\newpage\clearpage
\lthtmlfigureA{figure4825}%
\begin{figure}  \begin{center}
\par
\htmlimage{scale=2.0}
\par
\latex{
      \includegraphics[scale=0.5]{pictures/tut_bond_plot}
    }
    \end{center}
    
  \end{figure}%
\lthtmlfigureZ
\lthtmlcheckvsize\clearpage}

{\newpage\clearpage
\lthtmlinlinemathA{tex2html_wrap_inline5726}%
\fbox{
    \begin{minipage}{.2\textwidth}
      \includegraphics[width=2.3 cm, height=2.3 cm]{pictures/tut0_science}
    \end{minipage}
    \begin{minipage}[r]{.75\textwidth}
      \noindent\small\textsf{\textbf{Ubiquitin unfolding.} The plot displayed shows the length of the bond over time. You can think of it as a plot of the ``length'' of the molecule, as it is
pulled at constant force.\\Note that at the beginning of the curve,  the curve is flattening, like if the pulling did not affect the structure of the molecule, and suddenly, the distance increases a lot. Take a look
at the time this occurs: when the two yellow $\beta$~strands separated! This
means that, those H-bonds that keep the $\beta$~strands together provide an
initial resistance to the unfolding of the protein, and once they are
broken, the protein unfolds in a more regular way. Try identifying what happens in the second jump. You will have identified key features of ubiquitin unfolding!}
    \end{minipage}
  }%
\lthtmlinlinemathZ
\lthtmlcheckvsize\clearpage}

\stepcounter{subsection}
{\newpage\clearpage
\lthtmlinlinemathA{tex2html_wrap_inline5734}%
\fbox{
    \begin{minipage}{.2\textwidth}
      \includegraphics[width=2.3 cm, height=2.3 cm]{pictures/tut0_science}
    \end{minipage}
    \begin{minipage}[r]{.75\textwidth}
      \noindent\small\textsf{\textbf{Root Mean Square Deviation.} The Root Mean Squared Deviation (RMSD) is a numerical measure of the difference between two structures.  It is defined as : \\
  \begin{equation}
    RMSD = \sqrt{\frac{\sum_{i=1}^{N_{atoms}}{(r_{i}(t_{1})-r_{i}(t_{2}))^2}}{N_{atoms}}}
  \end{equation}\\
where $N_{atoms}$\  is the number of atoms whose positions are being compared, and $r_{i}(t)$\  is the position of atom $i$\  at time $t$\  (or, if you are comparing two molecules, like in Unit 2, $t$\  labels the molecules).}
    \end{minipage}
  }%
\lthtmlinlinemathZ
\lthtmlcheckvsize\clearpage}

\addtocounter{nitems}{1}
\addtocounter{nitems}{1}
\addtocounter{nitems}{1}
{\newpage\clearpage
\lthtmlinlinemathA{tex2html_wrap_inline5766}%
\fbox{
    \begin{minipage}{.2\textwidth}
      \includegraphics[width=2.3 cm, height=2.3 cm]{pictures/tut0_science}
    \end{minipage}
    \begin{minipage}[r]{.75\textwidth}
      \noindent\small\textsf{\textbf{Equilibration.} The trajectory you loaded is the equilibration of the ubiquitin
and water system. In reality, this trajectory was simulated before the
pulling trajectory you saw before. In MD, a system must be equilibrated
first to have the stability necessary to observe other features related to
the system and not to changes in the environment. \\\\Is there anything
you can say about this trajectory? Everything seems to be moving around
randomly. Note how the water comes closer to the protein filling in the
gaps, and the size of the water box decreases to obtain the right water
density. }
    \end{minipage}
  }%
\lthtmlinlinemathZ
\lthtmlcheckvsize\clearpage}

\addtocounter{nitems}{1}
{\newpage\clearpage
\lthtmlinlinemathA{tex2html_wrap_inline5156}%
$\leq$%
\lthtmlinlinemathZ
\lthtmlcheckvsize\clearpage}

{\newpage\clearpage
\lthtmlinlinemathA{tex2html_wrap_inline5158}%
$\}$%
\lthtmlinlinemathZ
\lthtmlcheckvsize\clearpage}

\addtocounter{nitems}{1}
\addtocounter{nitems}{1}
\addtocounter{nitems}{1}
{\newpage\clearpage
\lthtmlpictureA{tex2html_wrap5914}%

% latex2html id marker 5914
\framebox[\textwidth]{
    \begin{minipage}{.2\textwidth}
      \includegraphics[width=2.5 cm, height=1.6 cm]{pictures/tut0_infobox}
   \end{minipage}
    \begin{minipage}[r]{.75\textwidth}
      \noindent\small\textsf{\textbf{More RMSD.} You can try sourcing another script called {\tt 
    rmsd-fullthrottle.tcl} Take a look at this script to learn simple
  procedures in tcl, as well as calculating the RMSD of each residue over
  time and coloring residues according to their RMSD similarly that in Unit
  \ref{unit2}.}
    \end{minipage}
  }%
\lthtmlpictureZ
\lthtmlcheckvsize\clearpage}


\end{document}
